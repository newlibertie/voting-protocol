
\section{Introduction} \label{section:Introduction}
It can be said that we live in a democratic age.
By and large, ideas flow freely throughout the planet riding
on pictures, notes, videos, voice messages, emails, and so
many other media effortlessly over social networks blogs
websites discussion groups etc.
It can be safely said that this is the information age and
as people are freely able to exchange ideas, invariably the rulers
have to pay heed to the voices of the people.
This holds true not only for democracies where leaders are elected
through a popular (or other) vote but also to some degree for
what are called tyrannical regimes. The leaders have to acquiese to
the power of the people.  There are several examples~\cite{}

While it is true that we live in this democratic age, it can not be
denied that there are serious dangers or challenges lurking around
in the form of propaganda campaigns, fake news, and the coarsening
and poisoning of the discourse especially in  online medium like
Facebook where discussion may easily go out of control turning
well meaning friends or relatives into vicious political opponents.

The reason for this breakdown is relatively easy to understand in
as a dynamical system.
Face to face discussions and deliberations have a large non-verbal
component~\footnote{By some measures the majority of communication is
  non-verbal}.
As we have moved our political discourse to a non face-to-face setting,
we have lost the subtle cues that would provide real-time feedback
and keep us on track by understanding the other person's views and
legitimate concerns.
Without having this real-time common understanding to guide our 
discussion, it is very easy to fall into acrimony.
Arguments are not enough.  Lack of understanding is the challenge.

One may think that perhaps the solution is to
encourage more participative democracy.
While theoretically attractive, this approach
is impractical for a number of reasons. The total mental
investment that a person would make in political activity is
limited for most people.
Having access to (mostly comfortable) viewpoints seems to suffice
for most people.

As observed personally by the author,  the typical meeting and
discussion areas like coffee shops or bars are almost devoid
of even a trace of political discussion.  And where these discussions
happen, they are of extremely local nature among people with high level
of common interest.  The internet political discourse has eaten away
the capability of people to reason, understand, and arrive at
sensible position agreed to by large majorities of people.  This is a
problem because in the absense of a robustly unbiased mechanism
for calculating public opinions,
politicians (and even pollsters) may be incentivized to use
propaganda and fake news in order to 'build' public opinions
for their narrow personal benefits.

Just to take a long view.  About a hudred years ago in the United States.
There was a functioning democracy and elections that operated pretty much
the same way as they do today. At that time, the major topic of
political discussion was womens suffrage.  \marginpar{17m 16s}


\paragraph{Outline}
The remainder of this article is organized as follows.
Section~\ref{previous work} gives account of previous work.
Finally, Section~\ref{conclusions} gives the conclusions.

