
\section{Voting Protocol}

% eventually consider moving to contributions subsection if needed
The voting protocol closely follows the cryptographic
setup described in Schoenmaker et. al.'s~\cite{Cramer:1997:SOE:1754542.1754554}
multi authority election scheme.  Given the different threat model,
the threshold encryption multi authority features are eliminated
and replaced with a single pollster service.

% Threat model table?

This is justified on the grounds that neither preventing malicious voters from
voting nor defending against malicious authorities is a
goal of this solution.

The malicious voter handling aspect is left to the human level pollsters
and human authorities which can presumably scale according to the level
of interest in the specific poll.
Our protocol ensures that the set of voting parties, the subset of eligible
voters selected after ballot casting, and the decisions about
accepted votes are publicly verifiable.
Another aspect of differentiation is \emph{coercion-freeness}, which
is an important requirement for a trustable voting infrastucture,
as described in~\cite{Karlof:2005:CVP:1251398.1251401}.
This property follows from the fact that only the aggregate totals are ever
decrypted, thereby providing voters deniability about their vote,
which in turn makes the
election coercion-free by allowing the voter to lie about the vote
they had cast\footnote{Ideally one would add a legal clause at the bottom of the
  ballot stating that it is your right, and under some circumstances,
  perhaps your duty, to lie about your vote. Such a disclaimer
  would make the deniability based coercion freeness obvious to
  unsophisticated voters.  Achieving coercion freeness requires both social and
  technological actions.  Our protocol provides the latter.}.

Malicious authorities are not considered as a threat because we
propose to establish authority honesty via open source implementation as well
as massive public validation of the algorithms and artifacts involved.
The authority supports the fully open source voting protocol in an
open verifiable manner.  The voting protocol is explained below in terms of
its phases:

\subsection{Poll creation}
On receiving a request to create a poll, the Pollster service
chooses a large prime $p$, a generator $g$, and an integer private key $s < p$
large enough such that an $O(\sqrt s)$ time algorithm is computationally
infeasible.
The public key $h \equiv g^s\bmod p$ is computed from the private key.
The values $p$, $g$ and $h$
are exposed on the public bulletin board.

This step shall also disclose the non-cryptographic
attributes of the poll. For example, the expected number of voters,
the duration of the poll, as well as the
locations where a vote may be cast and the location of the
bulletin board where the results shall be available.

This stage can also be used to cryptographically commit the
polling parameters.  Accordingly the hash of the contents of the
initial bulletin board shall be stored on a blockchain
in a future version of our implementation.


\subsection{Casting encrypted ballots}

The specific values sent out in the encrypted ballots depends on
whether the vote is positive or negative.  The values of the variables
used in casting the universally verifiable ballot are shown in
Table~\ref{table:ballotparams}.  The values are expressed in terms of
a new generator $G$ and random numbers $\alpha$, $r_1$ and $r_2$.
This new generator is required in order to
provide the zero knowledge proof of validity of the ballot,
whereas generator $g$ was previously used for casting encrypted ballot.
% TODO check if the above is sufficient? G a new generator and if so why




\begin{table}[t]
\caption{Values of ballot parameters}
\begin{center}
  \begin{tabular}{c|c|c} 
    Variable & Vote $+1$ & Vote $-1$ \\\hline
    $x$   &  $g^\alpha$ &   $g^\alpha$ \\
    $y$   &  $h^\alpha G$ & $\frac{h^\alpha}{G}$  \\
    $a_1$ &  $g^{r_1}x^{d_1}$ & $g^\omega$  \\
    $b_1$ &  $h^{r_1}(yG)^{d_1}$ & $h^\omega$  \\
    $a_2$ &  $g^\omega$ &  $g^{r_2}x^{d_2}$ \\
    $b_2$ &  $h^\omega$ & $h^{r_2} \big( \frac{y}{G} \big)^{d_2}$ \\
    $r_1$ &    Random     &  $\omega - \alpha d_1$ \\
    $r_2$ &  $\omega - \alpha d_2$ & Random \\
\end{tabular}
\label{table:ballotparams}
\end{center}
\end{table}



% TODO : get fig from jcn and data from page in yellow book
The universally veriable ballot to be cast by the voter $v_i$
can be expressed as a 3-tuple $\{T,C,D\}$, where:
\begin{equation} \label{eq3}
\begin{split}
T & = \{ x, y, a_1, b_1, a_2, b_2 \} \\
C & = H( T || v_i ) \\    % subtle conceptual bug - v_i is voter not verifier but see intro
D & = \{d_1, d_2, r_1, r_2 \} 
\end{split}
\end{equation}

Any verifier needs to establish that i)
the vote itself is valid, i.e. is a $+1$ or $-1$, and ii) that the ElGamal
pair $(x,y) \equiv (g^\alpha, h^\alpha m)$ is properly constructed using the
publicly known values $g$ and $h$.  These properties follow from testing
the following on the encrypted ballot:

\begin{equation} \label{eq4}
\begin{split}
C & = d_1 + d_2  \\
a_1 & = g^{r_1} x^{d_1}  \\
b_1 & = h^{r_1} (yG)^{d_1} \\
a_2 & = g^{r_2} x^{d_2} \\
b_2 & = h^{r_2}(\frac{y}{G})^{d_2} \\
\end{split}
\end{equation}


\subsection{Counting ballots}

Consider the encrypted vote part of the ballot $(x,y) \equiv (g^\alpha, h^\alpha G^m)$
where $m \in {+1, -1}$.
The homomorphic property of the encryption allows us to take the product(modulo
the large prime) of the ElGamal pairs, and the resultant values can be decrypted to
arrive at the total.   Consider several encrypted ballots
$(x_i,y_i) \equiv (g^{r_i} , h^{r_i} G^{m_i})$, and let 
\[
(X,Y) \equiv (\prod x_i , \prod y_i) 
\]
Then it follows that:
\begin{equation} \label{eq5}
\begin{split}
  \frac{Y}{X^s} & = \frac{\prod h^{r_i} G^{m_i}}{ \prod (g^{r_i})^s } \\
  & =  \frac{\prod g^{sr_i} G^{m_i}}{ \prod (g^{sr_i}) } \\
  & = \prod G^{m_i} \\
   & = G^{\sum m_i}
\end{split}
\end{equation}
Hence the total vote count is given as $\log_G \frac{Y}{X^s}$, which can be
calculated by the pollster service as it knows the private key $s$ corresponding
to this poll.


%\bf{TODO : zkp of this total is correct}



\subsection{Verification}\marginpar{\bf{TODO move to end or appendix}}
Positive vote for test 3, i.e. $a_2 =^{?} g^{r_2} x^{d_2}$  :
\begin{equation}
  \begin{split}
    a_2 & = g^{r_2} x^{d_2} \\
    & = g^{\omega - \alpha d_2}  g^{\alpha d_2} \\
    & = g^{\omega}  \\
  \end{split}
\end{equation}

Negative vote for test 3:
\begin{equation}
  \begin{split}
    a_2 & = g^{r_2} x^{d_2} \\
    & = g^{r_2}  x^{d_2}\\   % TODO add guideling narrative like "by construction"
  \end{split}
\end{equation}

