

\section{Preliminaries}

Our solution can be described in terms of the participants,
the  notations used to describe the interactions,
and 
the primitives used to achieve the desired security
properties.

\subsection{Participants}
\paragraph{Pollster service}
The poll is run by a \emph{Pollster service} application which
allows users to interact with it using a RESTful
interface.  The pollster service implements the following functions:
 
\begin{itemize}
\item{ \textsc{Create poll\\}
  Lay users, or \emph{human pollsters}, can define polls.
  Upon poll creation the pollster service accepts ballots
  on a voting URL.  This voting URL can be shared with prospective voters.
}
\item{ \textsc{Accept ballot\\}
  Voters cast ballots using the voting URL RESTful interface exposed
  by the pollster service. Because the ballot needs to have
  a well defined structure with cryptographic contents, an
  accompanying library is supported to make it easy for lay voters to
  cast their votes.
}
\item{ \textsc{Bulletin board\\}
  The pollster service publishes ballots on a bulletin
  board.
  Once a sufficient number of ballots have been collected,
  and any disallowed voters trimmed, 
  a tally of valid votes are also published on the bulletin board.
}
\end{itemize}

\paragraph{Voters}
The voters are expected to be unsophisticated internet users who
cast their ballots through a \emph{votster widget}.  The function of this
widget is to faithfully convert the voter choice into a anonymity
preserving ballot and to post it on the \textsc{Accept ballot} interface
of the Pollster service.

\paragraph{Verifiers}
Verifiers may be lay users or sophisticated cryptographers who can
rerun the ballots-to-tally computation being exposed on the
\textsc{Bulletin board} interface of the the pollster service.


\subsection{Notations}

\begin{table}[t]
\caption{Notation}
\begin{center}
\begin{tabular}{|cc|} \hline
$K_A$   &  Public key of the participant $A$\\
$K^{-1}_A$   &  Private key of the participant $A$ \\
$\{x,y,z\}$   &  A message containing $x$, $y$ and $z$ \\
$\{x\}_A$ &  A message signed by $A$ \\ \hline
\end{tabular}
\label{table:notation}
\end{center}
\end{table}
  
The general notations used in this paper are shown in Table~\ref{table:notation}.
We consider $n$ voters $v_1\ldots v_n$  who want to vote on a
binary poll $\mathcal{B}$ with choices $0$ or $1$.
The pollster service is represented as $\mathbf{P}$.
It has a well known public key $K_P$ which
is used for creating digitally signed messages.

The participants communicate with each other using messages, which are
represented using the security protocol
notation~\footnote{ Even though an abstract
  security protocol notation is used for protocol
  specification and analysis, the implemented messages may be represented
  in a more machine friendly manner, for example as JSON objects.
}.
For example, $A$ sending a message containing
$X$ and $Y$ to $B$ is represented as follows:
\[
A \rightarrow B \hspace{10pt} : \hspace{10pt} \{X,Y\}  
\]

\subsection{ElGamal cryptosystem}
The ElGamal cryptosystem is a well known
and established public key
cryptosystem based on the Discrete Logarithms problem~\cite{1057074}.
Its security
comes from the assumption that computing discrete logarithms over
a prime group is hard.
\begin{assumption}
Given a large prime $p$ and a generator $g$,
it is computationally infeasible to find $x$ from $y \equiv g^x \bmod p$.
\end{assumption}

The ElGamal cryptosystem works by all players initially agreeing to a large prime $p$
and a generator $g$.  The receiver chooses a secret key $s < p$ and publishes
the public key $h \equiv g^s \bmod p$, which is assumed to be well-known to
all parties.

In the context of the ElGamal cryptosystem
we will assume all random choices and operations are done
modulo $p$.  Hence one can succintly represent the
public key publication message as
\[
\text{Receiver} \rightarrow \text{Sender} \hspace{10pt} : \hspace{10pt} \{g^s\}  
\]

Suppose the sender wants to send a message $m$ to the receiver, then it chooses
a random integer $r$ and sends the following message:
\[
\text{Sender} \rightarrow \text{Receiver} \hspace{10pt} : \hspace{10pt} \{g^r, h^rm\}
\]

Since the receiver does not know about the random choice $r$ made by the sender, we
represent the received message as $\{x, y\}$.  Using its secret $s$, the receiver recovers
the message by computing $\frac{y}{x^s}$, as shown below.
\begin{equation} \label{eq1}
\begin{split}
\frac{y}{x^s} & = \frac{h^rm}{g^{r^s}} \\
              & = \frac{{{(g^s)}^r}m}{g^{r^s}}\\
              & = m
\end{split}
\end{equation}

\subsection{Non-interactive zero knowledge proofs}

Zero knowledge proofs permit a prover to convince a verifier about their knowledge
of a particular information without divulging that information.  The proving
process can either take the form of an interactive session where the verifier
can issue challenges and the prover can respond correctly.  These ineractions can
be converted into an equivalent non-interactive proof,
thereby avoiding the need for participants
to be online.  One such construction is the Schnorr protocol, which 
provides universal verifiability.

The objective of the Schnorr protocol is for the prover to convince a
verifier that it knows the discrete logarithm $x$ of a well known
value $y \equiv g^x$ \emph{without} disclosing the value $x$.  The interactive
protocol proceeds by the prover and verifier selecting random numbers,
$e$ and $c$ respectively,
and exchanging the following messages:
\[
\begin{array}{lcccc} 
  \text{Prover} \rightarrow \text{Verifier} & & : & & \{ t \equiv g^e  \} \\
  \text{Verifier} \rightarrow \text{Prover} & & : & & \{ c \} \\
  \text{Prover} \rightarrow \text{Verifier} & & : & & \{ d \equiv e - xc \} 
\end{array}
\]
The verifier checks if $y^cg^d = t$, which follows from :
\begin{equation} \label{eq2}
\begin{split}
y^cg^d & = g^{xc}g^d \\
       & = g^{(xc + d)}  \\
       & = g^{(xc + e - xc)} \\
       & = g^e \\
       & = t 
\end{split}
\end{equation}

A non-interactive proof is generated by constraining the verifier challenge
$c$ to be predictable by the prover yet to resist an adaptive
ciphertext attack.
This can be done by having it depend on a secure hash function $H$ whose
output is sufficiently random and unpredictable.  Then the zero
knowledge interactive protocol with the verifier $v_i$
can be transformed into its non-interactive version with a
transcript as follows:
\[
\{ g^e , H(g^e, v_i) , e - xH(g^e, v_i) \} 
\]

% TODO : isnt it true we can have recursively contained nizkp s
% such that $v_i \equiv \star$ is universally verifiable, whereas
% org.* may be a more specific nizkp

Such a transcript can be produced by the prover in advance, without
requiring any action from the verifier.  This Schnorr protocol
construction is used for ensuring universal verifiability of several
properties of the voting protocol.
