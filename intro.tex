
\section{Introduction} \label{section:Introduction}

The rise of online political discourse has counter-intuitively
reduced the ability of people to reason, understand, and arrive at
sensible political positions supported by large majorities of people.
% TODO : improve focus of the remaining in paragraph
Even though social networks like Facebook and Twitter have been used for political purposes, their effectiveness has had serious shortcomings. Most often the discussions tend to be shallow, and favor one extreme view or another and thereby fails to represent the broad spectrum of political preferences. This flattening of discourse also has the side effect of silencing voices which may differ from the prevalent majority opinion because of fear of retribution.

Lacking a reliable universally trusted mechanism
for calculating the true public opinions 
is a problem.
It creates an incentive for 
politicians, news media,  and even pollsters to use
propaganda and fake news in order to \emph{build} public opinions
aligned with their own narrow interests.

\subsection{Solution approach}
\paragraph{Coercion freeness}
Solving the problem of self-censorship requires
the following guarantee of privacy:
even though the voters are well known, the choices made by them
remain private.
Coercion freeness goes further by ensuring that 
voters can not provide evidence about the choice they made.
This protects the privacy of voters by ensuring that an adversary can
not deduce the vote count on a subset of voters by
colluding with malicious voters.
A combination of systems security
methods along with a cryptographic voting protocol
are used to provide coercion-freeness in our
solution~\cite{Cramer:1997:SOE:1754542.1754554,Karlof:2005:CVP:1251398.1251401}.

\paragraph{Universal verifiability}
The problem of potential pollster bias is addressed by having
a public bulletin board where all the ballots cast in the poll
can be validated. The bulletin board also exposes a computation
on the homomorphically encrypted ballots such that the
poll result can be be verified by anyone.  This universal
verifiability of poll results addresses the possibility of the pollster
making up poll results other than those derived from the ballots.
Non-interactive zero knowledge proofs based on Schnorr
protocol~\cite{Cramer:1994:PPK:646759.705842}
are used to support universal verifiability.  These proofs
are constructed for encrypted ballots as well as for
the poll result computation. 
Having publicly accessible universally verifiable proofs ensures
that the disclosed ballots are correctly tallied.
The complementary problem of ballot hiding is handled by
having the voter and the pollster engage in digitally signed communication
during ballot casting.  Any divergence from the correct voting
protocol is provable and shared with the public.


\paragraph{Deferring voter eligibity decisions}
Secure voting has traditionally been associated with government
or other official elections.  Accordingly, prior research and implementations
have focussed on enabling official or government elections where
establishing voter eligibility is typically the first stage of the
process.  We believe this choice constrains the existing approaches, thereby
limiting their applicability.  Our approach is
to defer voter eligibility to human judgment.  The eligibility decisions are done after
the ballots have been cast and the eligibility choices are universally verifiabile.



\paragraph{Enabling large scale secure polling}
We propose to secure the polling mechanism while leaving the policy on voter eligibilty
to human pollsters.  Removing the identity insfrastructure as a first class
participant makes it easy for unsophisiticated pollsters set up secure polls. 
Using the previously described building blocks, we propose to allow
lay users to setup and run cryptographically secure polls over the world wide
web, and yet derive poll results which can pass serious scrutiny.
Eliminating the voter pre-verification requirement also makes our solution
suitable for \emph{automatic polls} where active human ballot casting can be eliminated entirely.
For example,
one may automatically calculate the
popularity of political positions on online social networks by having a
opinion classifier automatically feed its results into a virtual opinion poll.
Depending on the application area, an appropriate level of effort can be invested in
voter verification in order to operate at a suitable level of accuracy versus
convenience trade-off.  




%\paragraph{Outline}
%The remainder of this article is organized as follows.
%Section~\ref{previous work} gives account of previous work.
%Finally, Section~\ref{conclusions} gives the conclusions.

