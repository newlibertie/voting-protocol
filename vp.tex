\documentclass[12pt,draft]{article}
\usepackage{amsmath}

\newtheorem{assumption}{Assumption}

\title{ {\bf  Coercion-free Universally-verifiable Voting over the Web} }

\author{Vivek Pathak}
\date{\today}



\begin{document}

\maketitle

\begin{abstract}
  We support  coercion-free
  and universally-verfiable voting for lay users over the world-wide-web.
  Established cryptographic and systems
  security primitives are adapted and enhanced in order to
  achieve these objectives. 
  Our voter security model differs from prior research in that it delays and
  defers
  the ascertainment of voter legitimacy to a human poll manager.
  This eliminates the requirement of an identity infrastructure,
  thereby making it possible
  for lay users to run cryptographically secure polls on demand.
\end{abstract}


\section{Introduction} \label{section:Introduction}
It can be said that we live in a democratic age.
By and large, ideas flow freely throughout the planet riding
on pictures, notes, videos, voice messages, emails, and so
many other media effortlessly over social networks blogs
websites discussion groups etc.
It can be safely said that this is the information age and
as people are freely able to exchange ideas, invariably the rulers
have to pay heed to the voices of the people.
This holds true not only for democracies where leaders are elected
through a popular (or other) vote but also to some degree for
what are called tyrannical regimes. The leaders have to acquiese to
the power of the people.  There are several examples~\cite{}

While it is true that we live in this democratic age, it can not be
denied that there are serious dangers or challenges lurking around
in the form of propaganda campaigns, fake news, and the coarsening
and poisoning of the discourse especially in  online medium like
Facebook where discussion may easily go out of control turning
well meaning friends or relatives into vicious political opponents.

The reason for this breakdown is relatively easy to understand in
as a dynamical system.
Face to face discussions and deliberations have a large non-verbal
component~\footnote{By some measures the majority of communication is
  non-verbal}.
As we have moved our political discourse to a non face-to-face setting,
we have lost the subtle cues that would provide real-time feedback
and keep us on track by understanding the other person's views and
legitimate concerns.
Without having this real-time common understanding to guide our 
discussion, it is very easy to fall into acrimony.
Arguments are not enough.  Lack of understanding is the challenge.

One may think that perhaps the solution is to
encourage more participative democracy.
While theoretically attractive, this approach
is impractical for a number of reasons. The total mental
investment that a person would make in political activity is
limited for most people.
Having access to (mostly comfortable) viewpoints seems to suffice
for most people.

As observed personally by the author,  the typical meeting and
discussion areas like coffee shops or bars are almost devoid
of even a trace of political discussion.  And where these discussions
happen, they are of extremely local nature among people with high level
of common interest.  The internet political discourse has eaten away
the capability of people to reason, understand, and arrive at
sensible position agreed to by large majorities of people.  This is a
problem because in the absense of a robustly unbiased mechanism
for calculating public opinions,
politicians (and even pollsters) may be incentivized to use
propaganda and fake news in order to 'build' public opinions
for their narrow personal benefits.

Just to take a long view.  About a hudred years ago in the United States.
There was a functioning democracy and elections that operated pretty much
the same way as they do today. At that time, the major topic of
political discussion was womens suffrage.  \marginpar{17m 16s}


\paragraph{Outline}
The remainder of this article is organized as follows.
Section~\ref{previous work} gives account of previous work.
Finally, Section~\ref{conclusions} gives the conclusions.

 



\section{Preliminaries}

Our solution can be described in terms of the participants,
the  notations used to describe the interactions,
and 
the primitives used to achieve the desired security
properties.

\subsection{Participants}
\paragraph{Pollster service}
The poll is run by a \emph{Pollster service} application which
allows users to interact with it using a RESTful
interface.  The pollster service implements the following functions:
 
\begin{itemize}
\item{ \textsc{Create poll\\}
  Lay users, or \emph{human pollsters}, can define polls.
  Upon poll creation the pollster service accepts ballots
  on a voting URL.  This voting URL can be shared with prospective voters.
}
\item{ \textsc{Accept ballot\\}
  Voters cast ballots using the voting URL RESTful interface exposed
  by the pollster service. Because the ballot needs to have
  a well defined structure with cryptographic contents, an
  accompanying library is supported to make it easy for lay voters to
  cast their votes.
}
\item{ \textsc{Bulletin board\\}
  The pollster service publishes ballots on a bulletin
  board.
  Once a sufficient number of ballots have been collected,
  and any disallowed voters trimmed, 
  a tally of valid votes are also published on the bulletin board.
}
\end{itemize}

\paragraph{Voters}
The voters are expected to be unsophisticated internet users who
cast their ballots through a \emph{votster widget}.  The function of this
widget is to faithfully convert the voter choice into a anonymity
preserving ballot and to post it on the \textsc{Accept ballot} interface
of the Pollster service.

\paragraph{Verifiers}
Verifiers may be lay users or sophisticated cryptographers who can
rerun the ballots-to-tally computation being exposed on the
\textsc{Bulletin board} interface of the the pollster service.


\subsection{Notations}

\begin{table}[t]
\caption{Notation}
\begin{center}
\begin{tabular}{|cc|} \hline
$K_A$   &  Public key of the participant $A$\\
$K^{-1}_A$   &  Private key of the participant $A$ \\
$\{x,y,z\}$   &  A message containing $x$, $y$ and $z$ \\
$\{x\}_A$ &  A message signed by $A$ \\ \hline
\end{tabular}
\label{table:notation}
\end{center}
\end{table}
  
The general notations used in this paper are shown in Table~\ref{table:notation}.
We consider $n$ voters $v_1\ldots v_n$  who want to vote on a
binary poll $\mathcal{B}$ with choices $0$ or $1$.
The pollster service is represented as $\mathbf{P}$.
It has a well known public key $K_P$ which
is used for creating digitally signed messages.

The participants communicate with each other using messages, which are
represented using the security protocol
notation~\footnote{ Even though an abstract
  security protocol notation is used for protocol
  specification and analysis, the implemented messages may be represented
  in a more machine friendly manner, for example as JSON objects.
}.
For example, $A$ sending a message containing
$X$ and $Y$ to $B$ is represented as follows:
\[
A \rightarrow B \hspace{10pt} : \hspace{10pt} \{X,Y\}  
\]

\subsection{ElGamal cryptosystem}
The ElGamal cryptosystem is a well known
and established public key
cryptosystem based on the Discrete Logarithms problem~\cite{1057074}.
Its security
comes from the assumption that computing discrete logarithms over
a prime group is hard.
\begin{assumption}
Given a large prime $p$ and a generator $g$,
it is computationally infeasible to find $x$ from $y \equiv g^x \bmod p$.
\end{assumption}

The ElGamal cryptosystem works by all players initially agreeing to a large prime $p$
and a generator $g$.  The receiver chooses a secret key $s < p$ and publishes
the public key $h \equiv g^s \bmod p$, which is assumed to be well-known to
all parties.

In the context of the ElGamal cryptosystem
we will assume all random choices and operations are done
modulo $p$.  Hence one can succintly represent the
public key publication message as
\[
\text{Receiver} \rightarrow \text{Sender} \hspace{10pt} : \hspace{10pt} \{g^s\}  
\]

Suppose the sender wants to send a message $m$ to the receiver, then it chooses
a random integer $r$ and sends the following message:
\[
\text{Sender} \rightarrow \text{Receiver} \hspace{10pt} : \hspace{10pt} \{g^r, h^rm\}
\]

Since the receiver does not know about the random choice $r$ made by the sender, we
represent the received message as $\{x, y\}$.  Using its secret $s$, the receiver recovers
the message by computing $\frac{y}{x^s}$, as shown below.
\begin{equation} \label{eq1}
\begin{split}
\frac{y}{x^s} & = \frac{h^rm}{g^{r^s}} \\
              & = \frac{{{(g^s)}^r}m}{g^{r^s}}\\
              & = m
\end{split}
\end{equation}

\subsection{Non interactive zero knowledge proofs}

Zero knowledge proofs permit a prover to convince a verifier about their knowledge
of a particular information without divulging that information.  The proving
process can either take the form of an interactive session where the verifier
can issue challenges and the prover can respond correctly.  These ineractions can
be converted into an equivalent non-interactive proof, thereby avoiding the need for participants
to be online.  One such construction is the Schnorr protocol, which 
provides the universal verifiability of ballots and poll results.

The objective of the Schnorr protocol is for the prover to convince a
verifier that it knows the discrete logarithm $x$ of a well known
value $y \equiv g^x$ \emph{without} disclosing the value $x$.  The interactive
protocol proceeds by the prover and verifier selecting random numbers,
$e$ and $c$ respectively,
and exchanging the following messages:
\[
\begin{array}{lcccc} 
  \text{Prover} \rightarrow \text{Verifier} & & : & & \{ t \equiv g^e  \} \\
  \text{Verifier} \rightarrow \text{Prover} & & : & & \{ c \} \\
  \text{Prover} \rightarrow \text{Verifier} & & : & & \{ d \equiv e - xc \} 
\end{array}
\]
The verifier checks if $y^cg^d = t$, which follows from :
\begin{equation} \label{eq2}
\begin{split}
y^cg^d & = g^{xc}g^d \\
       & = g^{(xc + d)}  \\
       & = g^{(xc + e - xc)} \\
       & = g^e \\
       & = t 
\end{split}
\end{equation}

A non-interactive proof is generated by constraining the verifier challenge
$c$ to be predictable by the prover yet to resist an adaptive
ciphertext attack.
This can be done by having it depend on a secure hash function $H$ whose
output is sufficiently random and unpredictable.  Then the zero
knowledge interactive protocol with the verifier $v_i$
can be transformed into its non-interactive version with a
transcript as follows:
\[
\{ g^e , H(g^e, v_i) , e - xH(g^e, v_i) \} 
\]

Such a transcript can be produced by the prover in advance, without
requiring any action from the verifier.  This Schnorr protocol type
construction is used for ensuring universal verifiability of several
properties of the voting protocol.



\section{Voting Protocol}

% eventually consider moving to contributions subsection if needed
The voting protocol closely follows the cryptographic
setup described in Schoenmaker et. al.'s~\cite{Cramer:1997:SOE:1754542.1754554}
multi authority election scheme.  Given the different threat model,
the threshold encryption multi authority features are eliminated
and replaced with a single pollster service.

% Threat model table?

This is justified on the grounds that neither preventing malicious voters from
voting nor defending against malicious authorities is a
goal of this solution.

The malicious voter handling aspect is left to the human level pollsters
and human authorities which can presumably scale according to the level
of interest in the specific poll.
Our protocol ensures that the set of voting parties, the subset of eligible
voters selected after ballot casting, and the decisions about
accepted votes are publicly verifiable.
Another aspect of differentiation is \emph{coercion-freeness}, which
is an important requirement for a trustable voting infrastucture,
as described in~\cite{Karlof:2005:CVP:1251398.1251401}.
This property follows from the fact that only the aggregate totals are ever
decrypted, thereby providing voters deniability about their vote,
which in turn makes the
election coercion-free by allowing the voter to lie about the vote
they had cast\footnote{Ideally one would add a legal clause at the bottom of the
  ballot stating that it is your right, and under some circumstances,
  perhaps your duty, to lie about your vote. Such a disclaimer
  would make the deniability based coercion freeness obvious to
  unsophisticated voters.  Achieving coercion freeness requires both social and
  technological actions.  Our protocol provides the latter.}.

Malicious authorities are not considered as a threat because we
propose to establish authority honesty via open source implementation as well
as massive public validation of the algorithms and artifacts involved.
The authority supports the fully open source voting protocol in an
open verifiable manner.  The voting protocol is explained below in terms of
its phases:

\subsection{Poll creation}
On receiving a request to create a poll, the Pollster service
chooses a large prime $p$, a generator $g$, and an integer private key $s < p$
large enough such that an $O(\sqrt s)$ time algorithm is computationally
infeasible.
The public key $h \equiv g^s\bmod p$ is computed from the private key.
The values $p$, $g$ and $h$
are exposed on the public bulletin board.

This step shall also disclose the non-cryptographic
attributes of the poll. For example, the expected number of voters,
the duration of the poll, as well as the
locations where a vote may be cast and the location of the
bulletin board where the results shall be available.

This stage can also be used to cryptographically commit the
polling parameters.  Accordingly the hash of the contents of the
initial bulletin board shall be stored on a blockchain
in a future version of our implementation.


\subsection{Casting encrypted ballots}

The specific values sent out in the encrypted ballots depends on
whether the vote is positive or negative.  The values of the variables
used in casting the universally verifiable ballot are shown in
Table~\ref{table:ballotparams}.  The values are expressed in terms of
a new generator $G$ and random numbers $\alpha$, $r_1$ and $r_2$.
This new generator is required in order to
provide the zero knowledge proof of validity of the ballot,
whereas generator $g$ was previously used for casting encrypted ballot.
% TODO check if the above is sufficient? G a new generator and if so why




\begin{table}[t]
\caption{Values of ballot parameters}
\begin{center}
  \begin{tabular}{c|c|c} 
    Variable & Vote $+1$ & Vote $-1$ \\\hline
    $x$   &  $g^\alpha$ &   $g^\alpha$ \\
    $y$   &  $h^\alpha G$ & $\frac{h^\alpha}{G}$  \\
    $a_1$ &  $g^{r_1}x^{d_1}$ & $g^\omega$  \\
    $b_1$ &  $h^{r_1}(yG)^{d_1}$ & $h^\omega$  \\
    $a_2$ &  $g^\omega$ &  $g^{r_2}x^{d_2}$ \\
    $b_2$ &  $h^\omega$ & $h^{r_2} \big( \frac{y}{G} \big)^{d_2}$ \\
    $r_1$ &    Random     &  $\omega - \alpha d_1$ \\
    $r_2$ &  $\omega - \alpha d_2$ & Random \\
\end{tabular}
\label{table:ballotparams}
\end{center}
\end{table}



% TODO : get fig from jcn and data from page in yellow book
The universally veriable ballot to be cast by the voter $v_i$
can be expressed as a 3-tuple $\{T,C,D\}$, where:
\begin{equation} \label{eq3}
\begin{split}
T & = \{ x, y, a_1, b_1, a_2, b_2 \} \\
C & = H( T || v_i ) \\    % subtle conceptual bug - v_i is voter not verifier but see intro
D & = \{d_1, d_2, r_1, r_2 \} 
\end{split}
\end{equation}

Any verifier needs to establish that i)
the vote itself is valid, i.e. is a $+1$ or $-1$, and ii) that the ElGamal
pair $(x,y) \equiv (g^\alpha, h^\alpha m)$ is properly constructed using the
publicly known values $g$ and $h$.  These properties follow from testing
the following on the encrypted ballot:

\begin{equation} \label{eq4}
\begin{split}
C & = d_1 + d_2  \\
a_1 & = g^{r_1} x^{d_1}  \\
b_1 & = h^{r_1} (yG)^{d_1} \\
a_2 & = g^{r_2} x^{d_2} \\
b_2 & = h^{r_2}(\frac{y}{G})^{d_2} \\
\end{split}
\end{equation}


\subsection{Counting ballots}

Consider the encrypted vote part of the ballot $(x,y) \equiv (g^\alpha, h^\alpha G^m)$
where $m \in {+1, -1}$.
The homomorphic property of the encryption allows us to take the product(modulo
the large prime) of the ElGamal pairs, and the resultant values can be decrypted to
arrive at the total.   Consider several encrypted ballots
$(x_i,y_i) \equiv (g^{r_i} , h^{r_i} G^{m_i})$, and let 
\[
(X,Y) \equiv (\prod x_i , \prod y_i) 
\]
Then it follows that:
\begin{equation} \label{eq5}
\begin{split}
  \frac{Y}{X^s} & = \frac{\prod h^{r_i} G^{m_i}}{ \prod (g^{r_i})^s } \\
  & =  \frac{\prod g^{sr_i} G^{m_i}}{ \prod (g^{sr_i}) } \\
  & = \prod G^{m_i} \\
   & = G^{\sum m_i}
\end{split}
\end{equation}
Hence the total vote count is given as $\log_G \frac{Y}{X^s}$, which can be
calculated by the pollster service as it knows the private key $s$ corresponding
to this poll.


%\bf{TODO : zkp of this total is correct}



\subsection{Verification}
\bf{TODO : move to appendix}.  Positive vote for test 3, i.e. $a_2 = g^{r_2} x^{d_2}$ ? :
\begin{equation}
  \begin{split}
    a2 & = g^{r2} x^{d2} \\
    & = g^{\omega - \alpha d_2}  g^{\alpha d_2} \\
    & = g^{\omega}  \\
  \end{split}
\end{equation}

Negative vote for test 3:
\begin{equation}
  \begin{split}
    a2 & = g^{r2} x^{d2} \\
    & = g^{r_2}  x^{d2} \\
  \end{split}
\end{equation}



\section{Types of Polls}
\subsection{Multiple Choice}
\subsection{Instant Runoff}

% maybe others
% TODO : Types of polls (gather from the web document repo and find a way of including here }




\bibliographystyle{abbrv}
\bibliography{vp}

\end{document}



\section{Proofs}
{\bf TODO}

\section{Conclusions}\label{conclusions}
        {\bf TODO}
        

        
